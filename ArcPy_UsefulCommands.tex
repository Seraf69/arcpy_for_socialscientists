\documentclass[12pt]{article} 
\usepackage{amsfonts, amsmath, amssymb} 
\usepackage{dcolumn, multirow} 
\usepackage{setspace} 
\usepackage{epsfig, subfigure, subfloat, graphicx}
\usepackage{tabularx} 
\usepackage{anysize, indentfirst, setspace} 
\usepackage{verbatim, rotating, paralist}
\usepackage{latexsym} 
\usepackage{amsthm} 
\usepackage{fullpage} 
\usepackage{longtable} 
\usepackage{graphicx} 
\usepackage{mathabx} 
\usepackage{txfonts} 
\usepackage{amsfonts} 
\usepackage{parskip} 
\usepackage{stmaryrd} 
\usepackage{mathrsfs} 
\usepackage{dsfont} 
\usepackage{comment} 
\usepackage{url} 
\usepackage{rotating} 
\usepackage{appendix}
\usepackage{natbib} 
\usepackage{tablefootnote}
\usepackage{hyperref}



\usepackage[margin=2.5cm]{geometry}

\title{Useful ArcPy Commands and Code Snippets}
\author{Nick Eubank}
\date{\today}


\begin{document} 
\maketitle

\subsection*{Individual Functions}

\subsubsection*{Data Manipulation}
\begin{itemize}
	\item Create Geodatabase: CreatePersonalGDB\_management() 
	\item Delete things: Delete\_management()
	\item Import GPS Points: MakeXYEventLayer\_management() \\
	\emph{Creates a Layer -- frequently need to then copy into featureclass for storage}
	\item Copy Layer to FeatureClass / Move Featureclass: CopyFeatures\_management()
	\item Copy Raster to FeatureClass: CopyRaster\_management()
	\item Export Attribute Table:  ExportXYv\_stats() \\
	\emph{Note: Will not label columns, and will include some leading columns, so be careful with use!} 
	\item Create a layer from featureclass: MakeFeatureLayer\_management()	\\
	\emph{Some tools only act on layers, and selections only work for layers}
\end{itemize}

\subsubsection*{Projections and Coordinate Systems}
\begin{itemize}
	\item Define Projection or Coordinate System: DefineProjection\_management() \\
	\emph{This is for SETTING the coordinate system, not RE-PROJECTING into NEW coordinate system}
	\item Re-Project to New Projection or Coordinate System: Project\_management()
\end{itemize}

\subsubsection*{Working with Fields}
\begin{itemize}
	\item Add a field (prior to calculating): AddField\_management()
	\item Calculate Field: CalculateField\_management()
	\item Delete Field: DeleteField\_management()
\end{itemize}
	
\subsubsection*{Selections}
Selections can be applied to either one of the two types of layers in ArcPy (see Tutorial, Section 8 for a discussion of layers). 	\emph{Selections can only be applied to layers, not feature classes!}
\begin{itemize}
	\item Create a Feature Layer from featureclass: MakeFeatureLayer\_management()	\\
	\item Manipulate Selections in Layer: SelectLayerByAttribute\_management() \\
	\emph{Note this is a multifaceted function -- the second argument in function allows you to specify whether you're creating new selection, clearing selection, subsetting, etc.}
	\item Manipulate Selections by Location: SelectLayerByLocation\_management() 
\end{itemize}

\subsubsection*{Cartography}
You can also use ArcPy to manipulate Map Documents (ending in .mxd). This is useful for cartography, and has it's own set of tools. 

When working with cartography, one often also works with selections. However, this is an area I don't know well, so this is admittedly thin!
\begin{itemize}
	\item Create Map Document Variable: mapping.MapDocument(``file.mxd'') \\
	\emph{e.g. mxd = arcpy.mapping.MapDocument(``file.mxd'')}
	\item Set the extent of the Map Document (i.e. zoom) to extent of selected elements of a layer (called myLayer): 
	\begin{quotation}
	\begin{verbatim}
		lyrExtent=myLayer.getSelectedExtent()
		df.extent=lyrExtent
	\end{verbatim}		
	\end{quotation}
\end{itemize} 


\section*{Idioms and Code Snippets}
\subsection*{Create New Database, Delete Old if Present, Report Back}
\begin{verbatim}
# Target geodatabase name
Output="NewDatabaseName.gdb"
# Home Folder
HomeFolder ="Z:/Documents/VAM"

try:
    arcpy.Delete_management(Output,HomeFolder)
    print "Deleted Old Database"
except:
    # Output.gdb did not exist, pass
    print "Did not delete"
    pass
arcpy.CreateFileGDB_management(HomeFolder, Output, "CURRENT")
print "Created New Database"	
\end{verbatim}
\pagebreak
\subsection*{Cartography: Zoom to different constituencies and print PDFs of each constituency}
\begin{verbatim}

mxd = arcpy.mapping.MapDocument("pc_maps_forexport.mxd")
mxd.author="Nick Eubank"
#mxd.save()

# Now I want to draw out the first item in the current data frame -- 
so call for list then subscript for item "0" Of course I only have one 
data frame, but this precludes the possibility I'll get back a list 
instead of a single item.

df=arcpy.mapping.ListDataFrames(mxd, "Layers")[0]

# Call list of layers. If you add a [0] to the end, you get a single 
object. I just pull out the objects in the next few lines. My first 
layer (lyrList[0]) is the layer of PC locations. The second (lyrList[1]) 
is a constituency map.
lyrList=arcpy.mapping.ListLayers(mxd)
print lyrList
pcsLayer=lyrList[0]
constsLayer=lyrList[1]

# List of target constituencies

consts=["1206",  "1207",  "1303",  "2105",  "2201",  "2301",  "2304",  
"2305",  "2408",  "2505",  "3104",  "3304",  "3305",  "4102"]

for item in consts:
    selectorString="[const_no]= "
    print item
    arcpy.SelectLayerByAttribute_management(pcsLayer, "NEW_SELECTION", 
		selectorString+item)
    print "Selections Made"
    lyrExtent=pcsLayer.getSelectedExtent()
    print "LyrExtent Run"
    df.extent=lyrExtent
    print "DF extent set"
    fileNameA="Z:/Dropbox/Elections_Sierra_Leone/pc_maps/map_"
    fileNameB=".pdf"
    # Now clear selections so cleaner looking
    arcpy.SelectLayerByAttribute_management(pcsLayer, "CLEAR_SELECTION")
    arcpy.mapping.ExportToPDF(mxd, fileNameA + item + fileNameB)
    print "Exported"

print "Total Success!"

\end{verbatim}
\end{document}